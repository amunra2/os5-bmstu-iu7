\setcounter{page}{2}
\section*{Цель работы}

Знакомство со средством дизассемблирования Sourcer, получение дизассемблированного кода ядра операционной системы Windows на примере обработчика прерывания INT 8h в virtual mode – специальном режиме защищенного режима (32-разрядный режим работы), который эмулирует реальный режим работы  вычислительной системы на базе процессоров Intel.

\clearpage

\section*{Листинг кода}

\subsection*{1 Листинг sub\_1} 
\begin{lstlisting}[style={asm}]
    sub_1       proc    near
; Сохранение регистров DS, AX
020A:07B9  1E					push	ds
020A:07BA  50					push	ax

; Адрес 0040:0000 загружается в регистр DS
020A:07BB  B8 0040				mov	ax,40h
020A:07BE  8E D8				mov	ds,ax

; Загрузка младшего байта регистра EFLAGS в AH
020A:07C0  9F					lahf				; Load ah from flags

; Флаг DF == 0 и старший бит IOPL == 0, тогда 
020A:07C1  F7 06 0314 2400			test	word ptr ds:[314h],2400h	; (0040:0314=3200h)
020A:07C7  75 0C				jnz	loc_2			; Jump if not zero

; сброс флага прерывания IF,
020A:07C9  F0> 81 26 0314 FDFF	           lock	and	word ptr ds:[314h],0FDFFh	; (0040:0314=3200h)
020A:07D0			loc_1:						;  xref 020A:07D6

; Восстановление значений флагов
020A:07D0  9E					sahf				; Store ah into flags

; Восстановление значений регистров
020A:07D1  58					pop	ax
020A:07D2  1F					pop	ds
020A:07D3  EB 03				jmp	short loc_ret_3		; (07D8)
020A:07D5			loc_2:						;  xref 020A:07C7

; иначе запрет маскируемых прерываний командой cli
020A:07D5  FA					cli				; Disable interrupts
020A:07D6  EB F8				jmp	short loc_1		; (07D0)
    
020A:07D8			loc_ret_3:					;  xref 020A:07D3
020A:07D8  C3					retn
\end{lstlisting}


\subsection*{2 Листинг int8h} 
\begin{lstlisting}[style={asm}]
; Вызов sub_1
020A:0746  E8 0070				call	sub_1			; (07B9)

; Сохранение регистров
020A:0749  06					push	es
020A:074A  1E					push	ds
020A:074B  50					push	ax
020A:074C  52					push	dx

; Адрес 0040:0000 загружается в DS
020A:074D  B8 0040				mov	ax,40h
020A:0750  8E D8				mov	ds,ax

; Адрес 0000:0000 загружается в ES
020A:0752  33 C0				xor	ax,ax			; Zero register
020A:0754  8E C0				mov	es,ax

; Инкремент младшей части счетчика таймера
020A:0756  FF 06 006C				inc	word ptr ds:[6Ch]	; (0040:006C=86B8h)

; Если младшая часть счетчика == 0, тогда
020A:075A  75 04				jnz	loc_1			; Jump if not zero

; инкремент старшей части счетчика таймера
020A:075C  FF 06 006E				inc	word ptr ds:[6Eh]	; (0040:006E=13h)

; иначе
020A:0760			loc_1:						;  xref 020A:075A

; Проверка : прошло ли 24 часа (18h = 24) - 2 старших байта счетчика
020A:0760  83 3E 006E 18			cmp	word ptr ds:[6Eh],18h	; (0040:006E=13h)
020A:0765  75 15				jne	loc_2			; Jump if not equal

; Проверка : два малдших байта счетчика == 176 (0B0h = 176)
020A:0767  81 3E 006C 00B0			cmp	word ptr ds:[6Ch],0B0h	; (0040:006C=86B8h)
020A:076D  75 0D				jne	loc_2			; Jump if not equal

; Обнуление счетчика, так как прошел день
020A:076F  A3 006E				mov	word ptr ds:[6Eh],ax	; (0040:006E=13h)
020A:0772  A3 006C				mov	word ptr ds:[6Ch],ax	; (0040:006C=86B8h)

; Фиксируем, что прошел день - записывается единица
020A:0775  C6 06 0070 01			mov	byte ptr ds:[70h],1	; (0040:0070=0)
020A:077A  0C 08				or	al,8

; Декремент счетчика (пока моторчик дисковода не отключится)
020A:077C			loc_2:						;  xref 020A:0765, 076D
020A:077C  50					push	ax
020A:077D  FE 0E 0040				dec	byte ptr ds:[40h]	; (0040:0040=5Ch)

; Проверка : значение счетчитка == 0
; Если да, то установка флага отключения моторчика и посылка команды
; в порт на отключение моторчика
020A:0781  75 0B				jnz	loc_3			; Jump if not zero

; Установка флага отключения моторчика дисковода
020A:0783  80 26 003F F0			and	byte ptr ds:[3Fh],0F0h	; (0040:003F=0)

; Отправка в порт команды на отключение моторчика
020A:0788  B0 0C				mov	al,0Ch
020A:078A  BA 03F2				mov	dx,3F2h
020A:078D  EE					out	dx,al			; port 3F2h, dsk0 contrl output

; Проверка : разрешены ли маскируемые прерывания (PF == 1)
020A:078E			loc_3:						;  xref 020A:0781
020A:078E  58					pop	ax

; Проверяется 2 бит - отвечает за флаг PF
020A:078F  F7 06 0314 0004			test	word ptr ds:[314h],4	; (0040:0314=3200h)

; Вызов максируемых разрешен, то переход
020A:0795  75 0C				jnz	loc_4			; Jump if not zero

; Загрузка младшего байта регистра флагов в AH
020A:0797  9F					lahf				; Load ah from flags
020A:0798  86 E0				xchg	ah,al
020A:079A  50					push	ax

; иначе - косвенный вызов 1Ch (командой call)
020A:079B  26: FF 1E 0070			call	dword ptr es:[70h]	; (0000:0070=6ADh)
020A:07A0  EB 03				jmp	short loc_5		; (07A5)
020A:07A2  90					nop

; Вызов пользовательского прерывания по таймеру
020A:07A3			loc_4:						;  xref 020A:0795
020A:07A3  CD 1C				int	1Ch			; Timer break (call each 18.2ms)

; Сброс контроллера прерываний
020A:07A5			loc_5:						;  xref 020A:07A0
020A:07A5  E8 0011				call	sub_1			; (07B9)
020A:07A8  B0 20				mov	al,20h			; ' '
020A:07AA  E6 20				out	20h,al			; port 20h, 8259-1 int command
										;  al = 20h, end of interrupt

; Восстановление значения регистров
020A:07AC  5A					pop	dx
020A:07AD  58					pop	ax
020A:07AE  1F					pop	ds
020A:07AF  07					pop	es

; Переход по адресу 020A:064Ch
020A:07B0  E9 FE99				jmp	$-164h
; ...
020A:06AC  CF					iret				; Interrupt return
\end{lstlisting}

\clearpage

\section*{Схемы алгоритмов}

\subsection*{1 Схема sub\_1} 
\img{200mm}{sub_1.png}{Схема sub\_1}
\clearpage

\subsection*{2 Схема int8h} 
\img{220mm}{int8h_1.png}{Схема int8h - 1}
\clearpage
\img{180mm}{int8h_2.png}{Схема int8h - 2}
\clearpage
\img{220mm}{int8h_3.png}{Схема int8h - 3}